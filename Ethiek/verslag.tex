\documentclass[a4paper]{article}
\usepackage[english]{babel}
\usepackage[utf8x]{inputenc}
\usepackage{amsmath, epic, eepic, float, subfig, amsfonts, color, amsthm, textcomp, microtype, fullpage}
\usepackage{SIunits}
\usepackage[parfill]{parskip}
\usepackage[pdftex]{graphicx}
\usepackage{color}
\definecolor{bluekeywords}{rgb}{0.13,0.13,1}
\definecolor{greencomments}{rgb}{0,0.5,0}
\definecolor{redstrings}{rgb}{0.9,0,0}
\usepackage[font=small,format=plain,labelfont=bf,up,textfont=it,up]{caption}
\usepackage[linkcolor=black,urlcolor=blue,citecolor=black]{hyperref}
\hypersetup{colorlinks=true}
\newcommand{\HRule}{\rule{\linewidth}{0.5mm}}
\usepackage{listings}
\lstset{language=[Sharp]C,
  showspaces=false,
  showtabs=false,
  breaklines=true,
  showstringspaces=false,
  breakatwhitespace=true,
  escapeinside={(*@}{@*)},
  commentstyle=\color{greencomments},
  keywordstyle=\color{bluekeywords},
  stringstyle=\color{redstrings},
  basicstyle=\ttfamily
}
\usepackage[T1]{fontenc}
\usepackage{beramono}
\usepackage{helvet}
\renewcommand*{\familydefault}{\sfdefault}

\begin{document}
\begin{titlepage}
\begin{center}
\includegraphics[width=0.5\textwidth]{./logo.pdf}~\\[1cm]


\textsc{\Large Ethiek 1}\\[0.5cm]

\HRule \\[0.4cm]
{ \LARGE \bfseries Project empathie in de sociale sector}\\[0.4cm]
{\large \textit{'t Wit Huis, Loppem}}\\[0.2cm]

\HRule \\[1.5cm]

\begin{minipage}{0.4\textwidth}
\begin{flushleft} \large
\emph{door:}\\
Haroen \textsc{Viaene}\\

\end{flushleft}
\end{minipage}
\begin{minipage}{0.4\textwidth}
\begin{flushright} \large
\large{2$^{\text{de}}$ fase bachelor Elektronica-ICT}\\
\end{flushright}
\end{minipage}

\vfill

{\large 2015-2016}

\end{center}
\end{titlepage}

\newpage
\begin{abstract}

% \section{'t Wit Huis, Loppem}

% voorstelling van de instelling, de doelgroep, met contactpersoon, 1

\end{abstract}

\section{Verslag}

% per uur, concreet, drie bladzijden

\subsection{Maandag 14 december 2015}

\subsection{Voormiddag}

Ik ben toegekomen per fiets om iets na 8u 's morgens. De echte werkdag begon een halfuur later, terwijl ik daarop wacht, loop ik al eens rond met mijn vader, die me kennis doet maken met de bewoners. Nadat dat gebeurd was, ging ik terug naar de personeelsruimte, en vertrok mijn vader naar huis; hij moet vandaag niet werken.

Rond 8:30 begint de vergadering, waarvan ik uiteraard niets mag vertellen in dit verslag. Hier kreeg ik mijn taak voor de dag en de volgende dagen, namelijk deze morgen \emph{klusklub}, in de namiddag \emph{wandelen}, morgen \emph{muziek} en in de namiddag \emph{bewonersvergadering} (voor de meeste andere begeleiders is het dan vergadering waar ik niet aan meedoe). Volgende maandag is het hetzelfde programma als vandaag.

Ik volg deze morgen opvoeder Jürgen Tierlinkx

Om 13:12 was mijn werk gedaan, en heb ik het evaluatieblad afgegeven aan de groepschef van de instelling. Daarna heb ik mijn verslag van de voormiddag geschreven. Om 13:30 begint de vergadering die de namiddag doet opstarten.

\subsection{Namiddag}

\subsection{Dinsdag 15 december 2015}

\subsection{Maandag 21 december 2015}

\subsection{Ervaringen}

%


\end{document}